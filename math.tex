\documentclass[12pt]{article}
\usepackage{amsmath}
\usepackage{amssymb}
\usepackage{commath}
\usepackage{mathtools}

% For multipage align*
\allowdisplaybreaks

\title{Math Notes for Compound Interest}
\author{}
\date{}

\begin{document}
\maketitle

First let's calculate the debt after a time period $t$ given a principal $P$, an
interest rate $r$, a compound frequency $f$, and a cyclic payment $p$.
\begin{align*}
    a &= 1 + \frac{r}{n}\\
    D_0 &= P\\
    D_1 &= aP - p\\
    D_2 &= a(aP - p) - p\\
        &= a^2 P - p (a + 1)\\
        &\vdots\\
    D_{nt} &= a^{nt} P - p \sum_{i = 0}^{nt - 1} a^i
\end{align*}

Remember that we can find the solution to the finite exponential series as so
\begin{align*}
    x &= \sum_{i = 0}^{nt - 1} a^i\\
    ax &= \sum_{i = 1}^{nt} a^i\\
    ax - a^{nt} + 1 &= \sum_{i = 0}^{nt - 1} a^i\\
    ax - a^{nt} + 1 &= x\\
    (1 - a) x &= 1 - a^{nt}\\
    x &= \frac{1 - a^{nt}}{1 - a}
\end{align*}

Plugging this in gives us our debt calculation.
\begin{align*}
    D_{nt} &= a^{nt} P - p \frac{1 - a^{nt}}{1 - a}
\end{align*}

From here we can calculate the minimum payment needed to pay off all debts in
a certain time frame.
\begin{align*}
    0 &= D_{nt}\\
    0 &= a^{nt} P - p \frac{1 - a^{nt}}{1 - a}\\
    p &= \frac{a^{nt}}{\frac{1 - a^{nt}}{1 - a}} P\\
      &= \frac{a^{nt} (1 - a)}{1 - a^{nt}} P\\
      &= \frac{a^{nt} (1 - a)}{1 - a^{nt}} P
\end{align*}

Taking the limit of $t$ we can also find the minimum payment needed to maintain
debt.
\begin{align*}
    p_\text{min} &= \lim_{t \to \infty} \frac{a^{nt} (1 - a)}{1 - a^{nt}} P\\
                 &= (a - 1) P
\end{align*}

We can also take the derivative of $t$ with respect to $p$ to find the optimal
trade-off point.
\begin{align*}
    \dpd{t}{p} &= -\delta\\
    \frac{\frac{1}{p} - \frac{1}{(1 - a) P + p}}{n \ln{a}} &= -\delta\\
    \frac{1}{p} - \frac{1}{(1 - a) P + p} &= -n\delta \ln{a}\\
    \frac{(1 - a) P}{p ((1 - a) P + p)} &= -n\delta \ln{a}\\
    p ((1 - a) P + p) &= -\frac{(1 - a) P}{n\delta \ln{a}}\\
    p^2 + (1 - a) Pp + \frac{(1 - a) P}{n\delta \ln{a}} &= 0\\
    p &= \frac{-(1 - a) P \pm \sqrt{(1 - a)^2 P^2 - \frac{(1 - a) P}{n\delta \ln{a}}}}{2}\\
      &= \frac{-(1 - a) P + \sqrt{(1 - a)^2 P^2 - \frac{(1 - a) P}{n\delta \ln{a}}}}{2}\\
\end{align*}

We can also calculate how long it will take to pay off all debts.
\begin{align*}
    0 &= a^{nt} (1 - a) P - p (1 - a^{nt})\\
    p &= ((1 - a) P + p) a^{nt}\\
    a^{nt} &= \frac{p}{(1 - a) P + p}\\
    nt &= \frac{\ln{p} - \ln\del{(1 - a) P + p}}{\ln{a}}\\
    t &= \frac{\ln{p} - \ln\del{(1 - a) P + p}}{n \ln{a}}\\
\end{align*}

\end{document}

